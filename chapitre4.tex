\chapter{Implantation De Le Loi De Commande}
\addcontentsline{toc}{chapter}{Implantation De Le Loi De Commande}
 \section{Calcul D’un Observateur Minimal Identité}

 \subsection{Le modèle linéarisé est-il observable ?}
 
 
 \subsection{Calcul Du Nouveau Observateur identité }
 
 \subsection{Insertion Dans Notre  Schéma Simulink l’observateur sous la forme d’un bloc State-Space.}
 
 
 \subsection{Étude en boucle fermer  }
  \subsubsection{ Vérification que les états estimés convergent vers les états réels du système linéarisé
lorsque les hauteurs initiales sont non nulles.}
  \subsubsection{Évaluation la vitesse de convergence.}
  
  
 \subsection{Réalisation Du Bouclage Déterminé Précédemment En Utilisant les états estimés (cf. figure 4.1).} 
 
\begin{center}
\includegraphics[scale=0.5]{fig3.png}
\captionof{figure}{\textit{ Observateur.}}
\label{fig3} 
\end{center}
 
 
 \subsection{Évaluation De  L’influence des Conditions Initiales}
 
 
 \subsection{Tracer  Des 3 courbes}
  \subsubsection{Schéma Simulink de l’asservissement par retour de sortie}
  \subsubsection{Évolution de l’erreur d’estimation;}
  \subsubsection{Évolution de la consigne et de la sortie mesurée}
  
 \section{Bruit Sur La Sortie Mesurée}

  \subsection{Que constate-t-on sur la sortie du système et sur les états estimés pour le schéma pour le schéma de commande précédent?}
  \subsection{Que constate-t-on sur la sortie du système et sur les états estimés pour le schéma pour le schéma de commande précédent?}
   \subsubsection{Quel est sont effet } 
  
  \section{Expliquez le phénomène observé lors des deux dernières questions en vous basant sur une analyse fréquen-tielle des observateurs.} 