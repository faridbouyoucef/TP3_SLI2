\chapter{Analyse  Et Calcul  D’une Loi De Commande  Par Retour  D’état }
\chaptermark{Loi De Cmd Par Retour D'état}
\addcontentsline{toc}{chapter}{Analyse  Et Calcul  D’une Loi De Commande  Par Retour  D’état }

 \section{Le modèle linéarisé est-il commandable?}
 En calculant la matrice de commandabilité $Co$ tel que : \\\\
 $Co=[B\quad AB \quad A^{2}B]$\\
 et on trouve :\\\\
 
 $Co=\begin{bmatrix} 
64.9351 & -0.5975 & 0.0110 \\
0 & 0.5975 & -0.0173 \\
0 & 0 & 0.0064  
\end{bmatrix}$\\\\\\
 
On a alors le $rang(Co)=3$ est qui est égale a la dimention du système, d'ou le système est commandable.\\  
De même, on a vérifier avec MATLAB la commandabilité du système.  \hyperref[section1.1]{(voir Annexe 1)}\label{annexe1}\\
 
 
 
 
 \section{Calcule Des Valeurs Des Gains K et N Permettant De Remplir L’ensemble Des Conditions.}
  
On considère pour cette étape de synthèse de la loi de commande que l'ensemble des états est mesuré. la loi de commande devra répondre aux spécifications du cahier des charges suivant: \\\\

\begin{itemize}
\item La consigne, notée $w(t)$, est un échelon de niveau exprimé en mètre. On pose $w(t) = 0, 05U(t)$;
\item Le temps de réponse $t_{r}$ de la sortie doit respecter $t_{r} \leq 90 s$;
\item L’erreur doit être nulle en régime permanent;
\item Le système doit rester dans le cadre d’une étude linéaire, c’est-à-dire que la vanne ne doit pas saturer:
$0 \leq Q_{1}(t) \leq Q_{max}$ \cite{ref1}\\\\\\ .
\end{itemize} 
 
\begin{center}
\includegraphics[scale=0.7]{fig2.png}
\captionof{figure}{\textit{ Retour d’état.\cite{ref1}}}
\label{fig2} 
\end{center}

La loi de commande choisie est un retour d'état classique représenté sur la figure 3.1 \\\\


Pour calculer la valeur du Gain K on pose :\\

$K=[k_{1}\quad k_{2} \quad k_{3}]$\\\\
et on a : 
$det(\lambda I_{dim3}-(A-BK))=(\lambda-P_{1})(\lambda-P_{2})(\lambda-P_{3})$\\\\
Selon notre cahier des charges, on veut que le temps de réponse $t_{r}$ de la sortie doit respecter $t_{r} \leq 90s$ et on a $t_{r}=\frac{-3}{Rel(P)}$, donc la premier valeur propre soit : $P_{1}=\frac{-3}{90}=-0.0333$ et $P_{2}$, $P_{3}$ prennent deux valeur inférieure à $P_{1}$, et alors on va choisir que : $P_{1}=-0.11$ et $P_{2}=0.17$\\\\    

Alors : $P_{1}=-0.03333 \quad P_{2}=-0.11 \quad P_{3}=-0.17$ (les valeurs propres désirés)\\

D'ou :\\
$\lambda I_{dim3}-(A-BK)=\begin{bmatrix} 
\lambda+0.0092+649352k_{1} & -0.0092+649352k_{2} & +649352k_{3} \\
0 & \lambda & 0 \\
0 & 0 & \lambda  
\end{bmatrix} $\\\\
 
après avoir l'identification des deux déterminants on obtient alors :\\

$k_{1}=0.0035  ,\quad k_{2}=0.0170  ,\quad k_{3}=-0.0141  $, \\\\

D'ou : $K=[0.0035 \quad 0.0170  \quad -0.0141]$\\\\

De même, on a calculer la valeur de K en utilisant MATLAB \hyperref[section1.2]{(voir Annexe 2)}\label{annexe2}\\\\

Pour calculer la valeur du Gain N on a :\\\\

$N=1/(C*(-A+B*K)^{-1}*B*\varepsilon $ \quad avec : $\varepsilon=1$, car on veut que l'erreur doit être nulle en régime permanent. \\\\

et on obtient $N=0.0108$ \quad \hyperref[section1.2]{(voir Annexe 2)}\label{annexe2}\\\\

\section{Les tracés de $q_{1}(t)$ et de $h _{1}(t)$ ainsi que le schéma bloc.}




\begin{center}
\includegraphics[scale=0.7]{schemabloc1.PNG}
\captionof{figure}{\textit{Schéma simulink avec un retour d'état.}}
\label{fig1} 
\end{center} 

\begin{center}
\includegraphics[scale=0.4]{q1(t).PNG}
\captionof{figure}{\textit{le tracé de $q_{1}(t)$.}}
\label{fig1} 
\end{center} 

\begin{center}
\includegraphics[scale=0.4]{h1(t).PNG}
\captionof{figure}{\textit{$h_{1}(t)$.}}
\label{fig1} 
\end{center} 
    






